%% start of file `template.tex'.
%% Copyright 2006-2013 Xavier Danaux (xdanaux@gmail.com).
%
% This work may be distributed and/or modified under the
% conditions of the LaTeX Project Public License version 1.3c,
% available at http://www.latex-project.org/lppl/.
% to build: $ pdflatex <this file>

\documentclass[11pt,letterpaper,sans]{moderncv}        % possible options include font size ('10pt', '11pt' and '12pt'), paper size ('a4paper', 'letterpaper', 'a5paper', 'legalpaper', 'executivepaper' and 'landscape') and font family ('sans' and 'roman')

% moderncv themes
\moderncvstyle{classic}                             % style options are 'casual' (default), 'classic', 'oldstyle' and 'banking'
\moderncvcolor{green}                               % color options 'blue' (default), 'orange', 'green', 'red', 'purple', 'grey' and 'black'
\moderncvicons{awesome}

\renewcommand{\familydefault}{\sfdefault}         % to set the default font; use '\sfdefault' for the default sans serif font, '\rmdefault' for the default roman one, or any tex font name
%\nopagenumbers{}                                  % uncomment to suppress automatic page numbering for CVs longer than one page

% character encoding
\usepackage[utf8]{inputenc}                       % if you are not using xelatex ou lualatex, replace by the encoding you are using
%\usepackage{CJKutf8}                              % if you need to use CJK to typeset your resume in Chinese, Japanese or Korean

% adjust the page margins
\usepackage[scale=0.75]{geometry}
%\setlength{\hintscolumnwidth}{3cm}                % if you want to change the width of the column with the dates
%\setlength{\makecvtitlenamewidth}{10cm}           % for the 'classic' style, if you want to force the width allocated to your name and avoid line breaks. be careful though, the length is normally calculated to avoid any overlap with your personal info; use this at your own typographical risks...

%c C plus plus
\def\cplusplus{C\raisebox{0.5ex}{\tiny\textbf{++}}}
\def\myplus{\raisebox{0.5ex}{\tiny\textbf{+}}}

%sensitive info
\def\myaddress{1 Blank Street}
\def\mycity{Blank City, BL}
\def\myzip{00000}
\def\myphone{1 (555) 123-4567}
\def\myemail{ben@example.com}

% personal data
\name{Ben}{Deane}
\title{Principal Software Engineer}                               % optional, remove / comment the line if not wanted
\address{\myaddress}{\mycity{ }\myzip}%{country}% optional, remove / comment the line if not wanted; the "postcode city" and and "country" arguments can be omitted or provided empty
\phone[mobile]{\myplus\myphone}                   % optional, remove / comment the line if not wanted
%\phone[fixed]{\myplus1 (555) 123-4567}                    % optional, remove / comment the line if not wanted
%\phone[fax]{+3 (456) 789 012}                      % optional, remove / comment the line if not wanted
\email{\myemail}                               % optional, remove / comment the line if not wanted
\social[github]{elbeno}                  % optional, remove / comment the line if not wanted
\social[linkedin]{elbeno}                  % optional, remove / comment the line if not wanted
\social[twitter][twitter.com/ben_deane]{@ben\_deane}                  % optional, remove / comment the line if not wanted
\photo[64pt][0.4pt]{picture}                       % optional, remove / comment the line if not wanted; '64pt' is the height the picture must be resized to, 0.4pt is the thickness of the frame around it (put it to 0pt for no frame) and 'picture' is the name of the picture file
\quote{I geek out on \cplusplus, functional programming, languages, Linux, emacs, network programming, systems architecture, simplicity.}

% to show numerical labels in the bibliography (default is to show no labels); only useful if you make citations in your resume
%\makeatletter
%\renewcommand*{\bibliographyitemlabel}{\@biblabel{\arabic{enumiv}}}
%\makeatother
%\renewcommand*{\bibliographyitemlabel}{[\arabic{enumiv}]}% CONSIDER REPLACING THE ABOVE BY THIS

% bibliography with mutiple entries
%\usepackage{multibib}
%\newcites{book,misc}{{Books},{Others}}
%----------------------------------------------------------------------------------
%            content
%----------------------------------------------------------------------------------
\begin{document}
%\begin{CJK*}{UTF8}{gbsn}                          % to typeset your resume in Chinese using CJK
%-----       resume       ---------------------------------------------------------
\makecvtitle

\section{Experience}
%\subsection{Vocational}
\cventry{2014--current}{Principal Software Engineer}{\href{http://www.blizzard.com/}{Blizzard Entertainment}}{Irvine, CA}{}{As a Principal Software Engineer, I am both the technical lead for a team, and a strategic contributor to Blizzard's companywide engineering culture and knowledge.\newline{}%
\hfill{ }\newline{}%
As the lead of a team of 8 engineers (my ``day job''), I:%
\begin{itemize}
\item am responsible for shipping and support the \href{http://battle.net/app/}{Battle.net Desktop App}.
\item work with engineers and PMs to plan, write tech specs, and develop logistical and technical roadmaps for features.
\item do regular engineering work - architecting and implementing systems, code and API reviews and documentation, fixing bugs.
\end{itemize}
\hfill{ }\newline{}%
As a strategic contributor to Blizzard (my ``principal duties''), I:%
\begin{itemize}
\item run talks and workshops on engineering techniques and tools (e.g. Test Driven Development).
\item coordinate internal ``engineering lightning talks'', helping engineers to gain presentation skills and spread knowledge.
\item created and moderate two internal mailing lists (``Good Code'' and ``\cplusplus'').
\item co-lead Blizzard's \href{http://news.uci.edu/feature/ready-to-play/}{mentorship program for Computer Game Science students at UC Irvine}.
\end{itemize}
\hfill{ }\newline{}%
I often field questions from engineers, particularly about architecture, \cplusplus{}, and git usage.\newline{}}

\cventry{2010--2014}{Lead Software Engineer}{\href{http://www.blizzard.com/}{Blizzard Entertainment}}{Irvine, CA}{}{I led the team that architected and built the Battle.net service and client libraries that are currently used by \textit{Diablo III}, \textit{Hearthstone}, \textit{Overwatch}, and the Battle.net Desktop App.\newline{}%
\hfill{ }\newline{}%
In this role, I:%
\begin{itemize}
\item managed a team of 3-6 (it changed over time) engineers.
\item handled live ops for my team's Battle.net systems.
\item architected and implemented:
\begin{itemize}
\item the social systems and routing infrastructure of the Battle.net server cloud.
\item the client APIs used by the game teams.
\item queueing and game allocation among available game servers.
\item the cooperative matchmaking service for \textit{Diablo III}, a system that handles on the order of a million open games at one time.
\end{itemize}
\end{itemize}}

\cventry{2008--2010}{Lead Software Engineer}{\href{http://www.blizzard.com/}{Blizzard Entertainment}}{Irvine, CA}{}{I led the team (of 6) that architected and built the Battle.net client libraries that shipped with \textit{StarCraft II}, \textit{Heroes of the Storm}, and \textit{World of Warcraft}. Among other things, I personally built social systems, wrote the system to download and cache data on the client, and implemented an ActionScript-to-\cplusplus{} FFI and data marshaling system.\newline{}}

\cventry{2005--2008}{Lead Software Engineer}{\href{http://www.ea.com/locations/los-angeles}{Electronic Arts}}{Playa del Rey, CA}{}{I led a team building \textit{Tiberium}, a hybrid FPS/RTS in the \textit{Command \& Conquer} universe using the Unreal 3 engine, and developed on PC, Xbox 360, and PS3. Among other things I did extensive work on improving the lighting, and on threading and optimization. In October 2008, EA cancelled the project.\newline{}}

\cventry{2003--2005}{Senior Software Engineer}{\href{http://www.ea.com/locations/los-angeles}{Electronic Arts}}{Playa del Rey, CA}{}{I was lead of the subteam building the network game for \textit{Goldeneye: Rogue Agent}, an FPS that shipped on Xbox and PS2. I implemented the core of the network game, working with a remote team in Montréal that was building levels and adding features.\newline{}}

\cventry{2001--2003}{Software Engineer}{\href{http://www.kuju.com/}{Kuju Entertainment}}{Godalming, UK}{}{I gained console experience on PS2 at Kuju, acting as lead engineer during the finalling stage of \textit{Reign of Fire}, and then taking a role as network lead for \textit{Warhammer 40,000: Fire Warrior}, which was one of the first titles to support the PS2 broadband adaptor.\newline{}}

\cventry{1995--2001}{Software Engineer}{\href{https://en.wikipedia.org/wiki/Bullfrog_Productions}{Bullfrog/Electronic Arts}}{Guildford/Chertsey, UK}{}{As a jack-of-all-trades engineer, I tackled many tasks: gameplay, systems, tools, UI, networking, localization, audio. My specific area of expertise became networking.\newline{}}
\section{Conference Talks}
\cvitemwithcomment{CppCon 2015}{\href{https://www.youtube.com/watch?v=OPoZWnYIcP4}{\color{blue}{Testing Battle.net}}}{}
\cvitemwithcomment{CppCon 2016}{\href{https://www.youtube.com/watch?v=ojZbFIQSdl8}{\color{blue}{Using Types Effectively}}}{}
\cvitemwithcomment{CppCon 2016}{\href{https://www.youtube.com/watch?v=B6twozNPUoA}{\color{blue}{\texttt{std::accumulate}: Exploring an Algorithmic Empire}}}{}

\section{Miscellaneous skills}
\cvitemwithcomment{Expert}{\cplusplus, git}{}
\cvitemwithcomment{Intermediate}{Python, Haskell, JavaScript, Unix, Lisp, French, \LaTeX, maths}{}

\section{Education}
\cventry{1992--1995}{MA (Cantab) Computer Science}{}{}{}{Emmanuel College, University of Cambridge}  % arguments 3 to 6 can be left empty
%\cventry{year--year}{Degree}{Institution}{City}{\textit{Grade}}{Description}

\section{The Boring Stuff}
\cvitem{}{I am a UK citizen and a US Permanent Resident (``green card'' holder).\newline{}References available on request.}

%% \section{Miscellaneous skills}
%% \cvdoubleitem{category 1}{XXX, YYY, ZZZ}{category 4}{XXX, YYY, ZZZ}
%% \cvdoubleitem{category 2}{XXX, YYY, ZZZ}{category 5}{XXX, YYY, ZZZ}
%% \cvdoubleitem{category 3}{XXX, YYY, ZZZ}{category 6}{XXX, YYY, ZZZ}

%% \section{Gameography}
%% \cvitem{in beta}{Overwatch}
%% \cvitem{2015}{StarCraft II: Legacy of the Void}
%% \cvitem{2015}{Heroes of the Storm}
%% \cvitem{2014}{World of Warcraft: Warlords of Draenor}
%% \cvitem{2014}{Hearthstone: Heroes of Warcraft}
%% \cvitem{2013}{StarCraft II: Heart of the Swarm}
%% \cvitem{2012}{World of Warcraft: Mists of Pandaria}
%% \cvitem{2012}{Diablo III}
%% \cvitem{2010}{World of Warcraft: Cataclysm}
%% \cvitem{2010}{World of Warcraft: Wrath of the Lich King (patch 3.3.5+)}
%% \cvitem{2010}{StarCraft II: Wings of Liberty}
%% \cvitem{unreleased}{Tiberium}
%% \cvitem{2007}{Medal of Honor: Airborne}
%% \cvitem{2004}{Goldeneye: Rogue Agent}
%% \cvitem{2003}{Warhammer 40K: Fire Warrior}
%% \cvitem{2002}{Reign of Fire}
%% \cvitem{2001}{Harry Potter and the Philosopher's Stone}
%% \cvitem{2001}{Shogun: Total War - The Mongol Invasion}
%% \cvitem{2000}{Shogun: Total War}
%% \cvitem{1999}{Theme Park World}
%% \cvitem{1998}{Populous: The Beginning}
%% \cvitem{1997}{Dungeon Keeper}
%% \cvitem{1997}{Theme Hospital}
%% \cvitem{1996}{Syndicate Wars}

%% \section{Extra 1}
%% \cvlistitem{Item 1}
%% \cvlistitem{Item 2}
%% \cvlistitem{Item 3. This item is particularly long and therefore normally spans over several lines. Did you notice the indentation when the line wraps?}

%% \section{Extra 2}
%% \cvlistdoubleitem{Item 1}{Item 4}
%% \cvlistdoubleitem{Item 2}{Item 5\cite{book1}}
%% \cvlistdoubleitem{Item 3}{Item 6. Like item 3 in the single column list before, this item is particularly long to wrap over several lines.}

%% \section{References}
%% \begin{cvcolumns}
%%   \cvcolumn{Category 1}{\begin{itemize}\item Person 1\item Person 2\item Person 3\end{itemize}}
%%   \cvcolumn{Category 2}{Amongst others:\begin{itemize}\item Person 1, and\item Person 2\end{itemize}(more upon request)}
%%   \cvcolumn[0.5]{All the rest \& some more}{\textit{That} person, and \textbf{those} also (all available upon request).}
%% \end{cvcolumns}

% Publications from a BibTeX file without multibib
%  for numerical labels: \renewcommand{\bibliographyitemlabel}{\@biblabel{\arabic{enumiv}}}% CONSIDER MERGING WITH PREAMBLE PART
%  to redefine the heading string ("Publications"): \renewcommand{\refname}{Articles}
\nocite{*}
\bibliographystyle{plain}
\bibliography{publications}                        % 'publications' is the name of a BibTeX file

% Publications from a BibTeX file using the multibib package
%\section{Publications}
%\nocitebook{book1,book2}
%\bibliographystylebook{plain}
%\bibliographybook{publications}                   % 'publications' is the name of a BibTeX file
%\nocitemisc{misc1,misc2,misc3}
%\bibliographystylemisc{plain}
%\bibliographymisc{publications}                   % 'publications' is the name of a BibTeX file

\clearpage


%-----       letter       ---------------------------------------------------------

%% % recipient data
%% \recipient{Company Recruitment team}{Company, Inc.\\123 somestreet\\some city}
%% \date{January 1, 1970}
%% \opening{Dear Sir or Madam,}
%% \closing{Yours faithfully,}
%% \enclosure[Attached]{résumé}          % use an optional argument to use a string other than "Enclosure", or redefine \enclname
%% \makelettertitle

%% Lorem ipsum dolor sit amet, consectetur adipiscing elit. Duis ullamcorper neque sit amet lectus facilisis sed luctus nisl iaculis. Vivamus at neque arcu, sed tempor quam. Curabitur pharetra tincidunt tincidunt. Morbi volutpat feugiat mauris, quis tempor neque vehicula volutpat. Duis tristique justo vel massa fermentum accumsan. Mauris ante elit, feugiat vestibulum tempor eget, eleifend ac ipsum. Donec scelerisque lobortis ipsum eu vestibulum. Pellentesque vel massa at felis accumsan rhoncus.

%% Suspendisse commodo, massa eu congue tincidunt, elit mauris pellentesque orci, cursus tempor odio nisl euismod augue. Aliquam adipiscing nibh ut odio sodales et pulvinar tortor laoreet. Mauris a accumsan ligula. Class aptent taciti sociosqu ad litora torquent per conubia nostra, per inceptos himenaeos. Suspendisse vulputate sem vehicula ipsum varius nec tempus dui dapibus. Phasellus et est urna, ut auctor erat. Sed tincidunt odio id odio aliquam mattis. Donec sapien nulla, feugiat eget adipiscing sit amet, lacinia ut dolor. Phasellus tincidunt, leo a fringilla consectetur, felis diam aliquam urna, vitae aliquet lectus orci nec velit. Vivamus dapibus varius blandit.

%% Duis sit amet magna ante, at sodales diam. Aenean consectetur porta risus et sagittis. Ut interdum, enim varius pellentesque tincidunt, magna libero sodales tortor, ut fermentum nunc metus a ante. Vivamus odio leo, tincidunt eu luctus ut, sollicitudin sit amet metus. Nunc sed orci lectus. Ut sodales magna sed velit volutpat sit amet pulvinar diam venenatis.

%% Albert Einstein discovered that $e=mc^2$ in 1905.

%% \[ e=\lim_{n \to \infty} \left(1+\frac{1}{n}\right)^n \]

%% \makeletterclosing



%\clearpage\end{CJK*}                              % if you are typesetting your resume in Chinese using CJK; the \clearpage is required for fancyhdr to work correctly with CJK, though it kills the page numbering by making \lastpage undefined
\end{document}


%% end of file `template.tex'.
